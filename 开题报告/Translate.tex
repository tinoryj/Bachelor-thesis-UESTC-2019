\section{引言}

重复数据删除是许多云存储提供商和服务使用的一个过程,用于删除存储在云中的冗余数据副本。已经显示[12,14]其在实践中大大降低了存储要求,因为个人和公司的用户通常存储相同或相似的内容。重复数据删除可以在服务器端或客户端进行。在服务器端重复数据删除中,服务器会检查客户端上载的文件是否已经存储。如果是这样,服务器不会再次存储它,而是由客户端记录文件所有权,并允许客当前户端使用合适的索引访问该共享文件。服务器端重复数据删除实现了减少存储空间占用的目的,但仍然要求客户端上载它希望存储的每个文件。在客户端重复数据删除中,希望首先上载文件的用户检查文件是否已经存储在云中,例如通过将文件的散列发送到服务器,该服务器检查其存储的文件散列列表。如果文件已经存储,则客户端不会发送文件,但服务器允许客户端像以前一样访问共享文件。因此,除了降低存储要求之外,客户端重复数据删除还可以大大降低云存储的带宽需求。由于与存储成本相比,通信成本可能很高,因此客户端重复数据删除通常优于服务器端重复数据删除,这是出于经济原因。重复数据删除可以针对文件或关于块进行,但我们不会关注这种差异,因为本文中考虑的许多攻击和对策都可以应用于任何一种方法。

\subsection{加密重复数据删除}

尽管存储和带宽节省了很多,但重复数据删除至少会导致两个主要的安全和隐私问题,这导致了最近在安全重复数据删除方面的广泛工作[14]。第一个问题是,如果部署了语义安全的端到端加密,则无法进行重复数据删除。在密文不可区分的情况下,不具有解密密钥的云服务提供商(CSP)将无法确定两个密文是否对应于相同的明文。为了解决这个问题[1,3,5,8,9,18],已经提出了几种替代的加密形式,以各种方式从文件本身导出加密密钥。这些工作通常会对文件不可预测性或关键分布做出强有力的假设。第二个问题是客户端重复数据删除可以作为在不同攻击下泄露信息的辅助渠道[7]。本文重点介绍这些侧通道攻击。

哈尼克等人。 [7]确定了客户端重复数据删除中由于侧通道引起的三次攻击。这些攻击适用于跨用户方案,其中上载相同文件的不同用户将对其数据进行重复数据删除。所有攻击的基本思想是,一个用户可以通过接收显示文件是否先前上载的信号来获取有关另一个用户文件的信息。在一个例子中,有时称为工资攻击,对手试图在她与CSP存储的雇佣合同中学习Alice的私人数据(她的工资)。攻击者在模板文件上插入猜测并将其上传到Alice的相应文件所在的CSP。然后,重复数据删除信号的出现将允许对手推断出猜测的正确性。

确定了这些侧通道攻击后,Harnik等人。 [7]提出了一种对策,其中关于文件是否已经上传的信号被随机化隐藏。更具体地,对于每个文件,随机均匀地选择阈值,并且如果先前上载的数量满足或超过阈值,则仅通知用户不上载文件。与基本的客户端重复数据删除相比,这显然会增加所需的带宽。在服务器端重复数据删除中不会出现旁道,因为客户端总是上传文件并允许服务器决定是否进行重复数据删除。因此,随机阈值对策可视为客户端重复数据删除的效率与服务器端重复数据删除的安全性之间的折衷。

\subsection{主要贡献}

虽然Harnik等的缓解思路。 [7]已经在文献[10,17]中进行了讨论和开发,没有正式建模和分析基于阈值的解决方案来保护侧通道攻击。这预示着有机会正式比较不同解决方案的有效性。本文的目的是通过以下方式纠正这种情况:
\begin{itemize}
    \item 为旁道重复数据删除策略提供正式定义,包括对策的有效性的自然衡量标准;
    \item 确定优化带宽和安全性战略所需的条件;
    \item 描述所有战略所需的安全性和效率之间的权衡;
    \item 显示Harnik等人的原始提案。 [7]在一个自然安全措施中提供最佳防御。
\end{itemize}


在其他情况下,攻击者可以使用类似的旁道。在独立和同时的工作中,Ritzdorf等人。 [16]考虑在重复数据删除存储系统中泄漏给好奇的云提供商的信息,特别关注使用内容定义的分块作为分段机制引起的泄漏。他们凭经验证明,在目标文件的许多强烈假设下,即使加密密钥未知,云提供商也可以高概率地推断低熵文件的内容。这个攻击向量与本文讨论的问题相关,但它确实强调了在存在恶意客户端和服务器的情况下分析云存储安全性的严格要求。另一个密切相关的领域是缓存隐私攻击,例如A cs等人所考虑的那些。 [2]在命名数据网络中;我们相信我们的模型也可以应用于这种情况。

本文的其余部分安排如下。 第2节回顾了对云存储的侧通道攻击和一些现有的对策。第3节讨论了我们的安全模型和防御的最优性。第4节证明了我们关于安全性和效率的主要定理,同时描述了良好的解除策略 安全与效率之间的基本权衡。 在第5节中,我们将讨论我们的工作如何与其他对策和方法相关。

